\chapter{Graph}

\section{Fundamentals}
	\kactlimport{BellmanFord.h}
	\kactlimport{FloydWarshall.h}

\section{Network flow and Matching}
	\kactlimport{Dinic.h}
    \kactlimport{MinCost.h}
	\kactlimport{PushRelabel.h}
	\kactlimport{Blossom.h}
	\kactlimport{HopcroftKarp.h}
	\kactlimport{WeightedMatching.h}
    \subsection{Hall's Theorem}
        In bipartite graphs, there exists a perfect matching covering the entire side $X$
        if and only if for every subset $Y \subseteq X$,
        \[
        |Y| \le |N(Y)|,
        \]
        where $N(Y)$ denotes the set of neighbors of $Y$.

    \subsection{K\H{o}nig's Theorem}
        In a bipartite graph, the size of a Minimum Vertex Cover
        is equal to the size of a Maximum Matching.
        A Minimum Vertex Cover is a minimum set of vertices such that every edge
        of the graph has at least one endpoint in the set.

        As a consequence,
        \[
        n - \text{Maximum Matching} = \text{Maximum Independent Set},
        \]
        where a Maximum Independent Set is the largest set of vertices
        with no edges between them.

        \paragraph{Recovering the Minimum Vertex Cover}
        Given a maximum matching in a bipartite graph $(X, Y)$:
        \begin{itemize}
            \item Construct the residual graph by orienting:
            \begin{itemize}
                \item non-matching edges from $X$ to $Y$;
                \item matching edges from $Y$ to $X$.
            \end{itemize}
            \item Perform a BFS or DFS starting from all free (unmatched) vertices in $X$.
            \item Let $Z_X$ be the set of reachable vertices in $X$,
                and $Z_Y$ the set of reachable vertices in $Y$.
        \end{itemize}

        The Minimum Vertex Cover is given by:
        \[
        (X \setminus Z_X) \cup Z_Y.
        \]

    \subsection{Node-Disjoint Path Cover}
        A node-disjoint path cover is a set of paths such that each vertex
        belongs to exactly one path.

        In a directed acyclic graph (DAG),
        \[
        \text{Minimum Node-Disjoint Path Cover} = n - \text{Maximum Matching}.
        \]

        The construction is as follows:
        for each vertex $u$, create a copy $u'$.
        Add an edge $u \to v'$ if there exists an edge $u \to v$ in the original graph.

        \paragraph{Recovering the Paths}
        \begin{itemize}
            \item Vertices that do not appear as destinations in the matching
                are starting points of paths.
            \item Each matching edge $u \to v'$ corresponds to an edge $u \to v$
                in the original DAG.
            \item Following these edges reconstructs all paths of the path cover.
        \end{itemize}

    \subsection{General Path Cover}
        A general path cover is a path cover where a vertex may belong
        to more than one path.

        In a DAG, the construction is similar to the node-disjoint case,
        but an edge $u \to v'$ exists if there is a path from $u$ to $v$
        in the original graph.

        \paragraph{Recovering the Cover}
        The vertices can be grouped according to the edges used in the matching
        to form the path cover.

    \subsection{Dilworth's Theorem}
        An antichain is a set of vertices such that there is no path
        between any pair of vertices in the set.

        In a directed acyclic graph,
        \[
        \text{Minimum General Path Cover} = \text{Maximum Antichain}.
        \]

        \paragraph{Recovering a Maximum Antichain}
        Given a minimum general path cover, selecting one vertex from each path
        produces a maximum antichain.


\section{DFS algorithms}
    \kactlimport{Bridges.h}
    \kactlimport{EulerPath.h}
	\kactlimport{SCC.h}
	\kactlimport{TwoSat.h}
	

% \section{Coloring}
% 	\kactlimport{EdgeColoring.h}

\section{Heuristics}
	\kactlimport{MaxClique.h}

\section{Trees}
	\kactlimport{Centroid.h}
	\kactlimport{HLD.h}
	\kactlimport{LCA.h}
	\kactlimport{VirtualTree.h}


\section{Math}
	\subsection{Number of Spanning Trees}
		% I.e. matrix-tree theorem.
		% Source: https://en.wikipedia.org/wiki/Kirchhoff%27s_theorem
		% Test: stress-tests/graph/matrix-tree.cpp
		Create an $N\times N$ matrix \texttt{mat}, and for each edge $a \rightarrow b \in G$, do
		\texttt{mat[a][b]--, mat[b][b]++} (and \texttt{mat[b][a]--, mat[a][a]++} if $G$ is undirected).
		Remove the $i$th row and column and take the determinant; this yields the number of directed spanning trees rooted at $i$
		(if $G$ is undirected, remove any row/column).

	\subsection{Erdős–Gallai theorem}
		% Source: https://en.wikipedia.org/wiki/Erd%C5%91s%E2%80%93Gallai_theorem
		% Test: stress-tests/graph/erdos-gallai.cpp
		A simple graph with node degrees $d_1 \ge \dots \ge d_n$ exists iff $d_1 + \dots + d_n$ is even and for every $k = 1\dots n$,
		\[ \sum _{i=1}^{k}d_{i}\leq k(k-1)+\sum _{i=k+1}^{n}\min(d_{i},k). \]

\section{Planar Graphs}
If $G$ has $k$ connected components, then $n - m + f = k + 1$.