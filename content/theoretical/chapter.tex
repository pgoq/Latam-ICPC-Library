\chapter{Theoretical}

\

\section{Mathematics}


\subsection{Recurrences}
If $a_n = c_1 a_{n-1} + \dots + c_k a_{n-k}$, and $r_1, \dots, r_k$ are distinct roots of $x^k - c_1 x^{k-1} - \dots - c_k$, there are $d_1, \dots, d_k$ s.t.
\[a_n = d_1r_1^n + \dots + d_kr_k^n. \]
Non-distinct roots $r$ become polynomial factors, e.g. $a_n = (d_1n + d_2)r^n$.

\subsection{Trigonometry}
\begin{align*}
\sin(v+w)&{}=\sin v\cos w+\cos v\sin w\\
\cos(v+w)&{}=\cos v\cos w-\sin v\sin w\\
\end{align*}
\begin{align*}
\tan(v+w)&{}=\dfrac{\tan v+\tan w}{1-\tan v\tan w}\\
\sin v+\sin w&{}=2\sin\dfrac{v+w}{2}\cos\dfrac{v-w}{2}\\
\cos v+\cos w&{}=2\cos\dfrac{v+w}{2}\cos\dfrac{v-w}{2}
\end{align*}
\[ (V+W)\tan(v-w)/2{}=(V-W)\tan(v+w)/2 \]
where $V, W$ are lengths of sides opposite angles $v, w$.
\begin{align*}
	a\cos x+b\sin x&=r\cos(x-\phi)\\
	a\sin x+b\cos x&=r\sin(x+\phi)
\end{align*}
where $r=\sqrt{a^2+b^2}, \phi=\operatorname{atan2}(b,a)$.

\subsection{Geometry}

\subsubsection{Triangles}
Side lengths: $a,b,c$\\
Semiperimeter: $p=\dfrac{a+b+c}{2}$\\
Area: $A=\sqrt{p(p-a)(p-b)(p-c)}$\\
Circumradius: $R=\dfrac{abc}{4A}$\\
Inradius: $r=\dfrac{A}{p}$\\
Length of median (divides triangle into two equal-area triangles): $m_a=\tfrac{1}{2}\sqrt{2b^2+2c^2-a^2}$\\
Length of bisector (divides angles in two): $s_a=\sqrt{bc\left[1-\left(\dfrac{a}{b+c}\right)^2\right]}$\\
Law of sines: $\dfrac{\sin\alpha}{a}=\dfrac{\sin\beta}{b}=\dfrac{\sin\gamma}{c}=\dfrac{1}{2R}$\\
Law of cosines: $a^2=b^2+c^2-2bc\cos\alpha$\\
Law of tangents: $\dfrac{a+b}{a-b}=\dfrac{\tan\dfrac{\alpha+\beta}{2}}{\tan\dfrac{\alpha-\beta}{2}}$\\

\subsubsection{Quadrilaterals}
With side lengths $a,b,c,d$, diagonals $e, f$, diagonals angle $\theta$, area $A$ and
magic flux $F=b^2+d^2-a^2-c^2$:

\[ 4A = 2ef \cdot \sin\theta = F\tan\theta = \sqrt{4e^2f^2-F^2} \]

 For cyclic quadrilaterals the sum of opposite angles is $180^\circ$,
$ef = ac + bd$, and $A = \sqrt{(p-a)(p-b)(p-c)(p-d)}$.

\subsubsection{Spherical coordinates}
\begin{center}
\includegraphics[width=25mm]{content/math/sphericalCoordinates}
\end{center}
\[\begin{array}{cc}
x = r\sin\theta\cos\phi & r = \sqrt{x^2+y^2+z^2}\\
y = r\sin\theta\sin\phi & \theta = \textrm{acos}(z/\sqrt{x^2+y^2+z^2})\\
z = r\cos\theta & \phi = \textrm{atan2}(y,x)
\end{array}\]

\subsubsection{Pick's Theorem}

The area of a simple polygon whose vertices have integer coordinates is:

\[ A = I + \frac{B}{2} - 1 \]

where $I$ is the number of interior integer points, and $B$ is the number of integer points in the border of the polygon.

\subsubsection{Two Ears Theorem}

Every simple polygon with more than 3 vertices has at least two non-overlapping ears (a ear is a vertex whose diagonal induced by its neighbors which lies strictly inside the polygon). Equivalently, every simple polygon can be triangulated.

\subsection{Derivatives/Integrals}
\begin{align*}
	\dfrac{d}{dx}\arcsin x = \dfrac{1}{\sqrt{1-x^2}} &&& \dfrac{d}{dx}\arccos x = -\dfrac{1}{\sqrt{1-x^2}} \\
	\dfrac{d}{dx}\tan x = 1+\tan^2 x &&& \dfrac{d}{dx}\arctan x = \dfrac{1}{1+x^2} \\
	\int\tan ax = -\dfrac{\ln|\cos ax|}{a} &&& \int x\sin ax = \dfrac{\sin ax-ax \cos ax}{a^2} \\
	\int e^{-x^2} = \frac{\sqrt \pi}{2} \text{erf}(x) &&& \int xe^{ax}dx = \frac{e^{ax}}{a^2}(ax-1)
\end{align*}

Integration by parts:
\[\int_a^bf(x)g(x)dx = [F(x)g(x)]_a^b-\int_a^bF(x)g'(x)dx\]

\subsection{Sums}
\[ c^a + c^{a+1} + \dots + c^{b} = \frac{c^{b+1} - c^a}{c-1},\; c\neq 1 \]

\[
\begin{aligned}
    1^2 + 2^2 + \dots + n^2 &= \frac{n(2n+1)(n+1)}{6} \\
    1^3 + 2^3 + \dots + n^3 &= \frac{n^2(n+1)^2}{4} \\
    1^4 + 2^4 + \dots + n^4 &= \frac{n(n+1)(2n+1)(3n^2+3n-1)}{30}
\end{aligned}
\]

\[ \sum_{i=0}^{n} ic^i = \frac{nc^{n+2} - (n+1)c^{n+1} + c}{(c-1)^2},\; c\neq 1 \]

\[ g_k(n) = \sum_{i=1}^n i^k = \frac{1}{k+1} \left( n^{k+1} + \sum_{j=1}^k \binom{k+1}{j+1} (-1)^{j+1} g_{k-j}(n) \right) \]

\subsection{Series}
$$e^x = 1+x+\frac{x^2}{2!}+\frac{x^3}{3!}+\dots,\,(-\infty<x<\infty)$$
$$\ln(1+x) = x-\frac{x^2}{2}+\frac{x^3}{3}-\frac{x^4}{4}+\dots,\,(-1<x\leq1)$$
$$\sqrt{1+x} = 1+\frac{x}{2}-\frac{x^2}{8}+\frac{2x^3}{32}-\frac{5x^4}{128}+\dots,\,(-1\leq x\leq1)$$
$$\sin x = x-\frac{x^3}{3!}+\frac{x^5}{5!}-\frac{x^7}{7!}+\dots,\,(-\infty<x<\infty)$$
$$\cos x = 1-\frac{x^2}{2!}+\frac{x^4}{4!}-\frac{x^6}{6!}+\dots,\,(-\infty<x<\infty)$$
$$\sum_{i=0}^{\infty} ic^{i} = \frac{c}{(1-c)^{2}}, \quad |c| < 1$$
$$(1+x)^n = \sum_{i=0}^{n} \binom{n}{i} x^i$$
$$\frac{1}{1-x} = \sum_{i=0}^{\infty} x^i,\,(-1 < x < 1)$$
$$\frac{1}{(1-x)^n} = \sum_{i=0}^{\infty} \binom{n+i-1}{n-1} x^i,\,(-1 < x < 1)$$


\subsection{Probability theory}
Let $X$ be a discrete random variable with probability $p_X(x)$ of assuming the value $x$. It will then have an expected value (mean) $\mu=\mathbb{E}(X)=\sum_xxp_X(x)$ and variance $\sigma^2=V(X)=\mathbb{E}(X^2)-(\mathbb{E}(X))^2=\sum_x(x-\mathbb{E}(X))^2p_X(x)$ where $\sigma$ is the standard deviation. If $X$ is instead continuous it will have a probability density function $f_X(x)$ and the sums above will instead be integrals with $p_X(x)$ replaced by $f_X(x)$.

Expectation is linear:
\[\mathbb{E}(aX+bY) = a\mathbb{E}(X)+b\mathbb{E}(Y)\]
For independent $X$ and $Y$, \[V(aX+bY) = a^2V(X)+b^2V(Y).\]

\subsubsection{Binomial distribution}
The number of successes in $n$ independent yes/no experiments, each which yields success with probability $p$ is $\textrm{Bin}(n,p),\,n=1,2,\dots,\, 0\leq p\leq1$.
\[p(k)=\binom{n}{k}p^k(1-p)^{n-k}\]
\[\mu = np,\,\sigma^2=np(1-p)\]
$\textrm{Bin}(n,p)$ is approximately $\textrm{Po}(np)$ for small $p$.

\subsubsection{First success distribution}
The number of trials needed to get the first success in independent yes/no experiments, each which yields success with probability $p$ is $\textrm{Fs}(p),\,0\leq p\leq1$.
\[p(k)=p(1-p)^{k-1},\,k=1,2,\dots\]
\[\mu = \frac1p,\,\sigma^2=\frac{1-p}{p^2}\]

\subsubsection{Poisson distribution}
The number of events occurring in a fixed period of time $t$ if these events occur with a known average rate $\kappa$ and independently of the time since the last event is $\textrm{Po}(\lambda),\,\lambda=t\kappa$.
\[p(k)=e^{-\lambda}\frac{\lambda^k}{k!}, k=0,1,2,\dots\]
\[\mu=\lambda,\,\sigma^2=\lambda\]

\section{Combinatorial}

\subsection{Binomial Identities}
\vspace{0.5em} 
$$
\begin{array}{cc}
\binom{n-1}{k} - \binom{n-1}{k-1} = \frac{n - 2k}{k} \binom{n}{k} &
\binom{n}{h}\binom{n-h}{k} = \binom{n}{k}\binom{n-k}{h}
\\[10pt]
\sum_{k=0}^{n} k \binom{n}{k} = n 2^{n-1} &
\sum_{k = 0}^n k^2 \binom{n}{k} = (n + n^2)2^{n-2}
\\[10pt]
\sum_{j = 0}^k\binom{m}{j} \binom{n-m}{k-j} = \binom{n}{k} &
\sum_{j = 0}^m \binom{m}{j}^2 = \binom{2m}{m}
\\[10pt]
\sum_{m = 0}^n \binom{m}{j} \binom{n-m}{k-j} = \binom{n+1}{k+1} &
\sum_{m = 0}^n \binom{m}{k} = \binom{n+1}{k+1}
\\[10pt]
\sum_{r = 0}^m \binom{n+r}{r} = \binom{n+m+1}{m} &
\sum_{k=0}^{n} \binom{n-k}{k} = \text{Fib}(n+1)
\\[10pt]
\sum_{k = 0}^n \binom{r}{k} \binom{s}{n - k} = \binom{r+s}{n}
\end{array}
$$


\subsection{Permutations}
	\subsubsection{Factorial}
		\import{factorial.tex}

	\subsubsection{Cycles}
		Let $g_S(n)$ be the number of $n$-permutations whose cycle lengths all belong to the set $S$. Then
		$$\sum_{n=0} ^\infty g_S(n) \frac{x^n}{n!} = \exp\left(\sum_{n\in S} \frac{x^n} {n} \right)$$

	\subsubsection{Derangements}
		Permutations of a set such that none of the elements appear in their original position.
		\[ \mkern-2mu D(n) = (n-1)(D(n-1)+D(n-2)) = n D(n-1)+(-1)^n = \left\lfloor\frac{n!}{e}\right\rceil \]

	\subsubsection{Burnside's lemma}
		 Counts the number of distinct colorings of an object under symmetry.
		 \[ {\frac {1}{|G|}}\sum _{{g\in G}}k^{\mathrm{cyc}(g)}, \]
		 where $G$ is the symmetry group, $k$ the number of colors, and $\mathrm{cyc}(g)$ the number of cycles induced by $g$.

		 Example: number of ways to color a necklace with $n$ beads using $k$ colors (rotations only):
		 \[ g(n) = \frac 1 n \sum_{i=0}^{n-1}{k^{(\text{gcd}(n, i))}} \]
		 where rotation $i$ shifts the necklace by $i$ positions.

\subsection{Partitions and subsets}
	\subsubsection{Partition function}
		Number of ways of writing $n$ as a sum of positive integers, disregarding the order of the summands.
		\[ p(0) = 1,\ p(n) = \sum_{k \in \mathbb Z \setminus \{0\}}{(-1)^{k+1} p(n - k(3k-1) / 2)} \]
		\[ p(n) \sim 0.145 / n \cdot \exp(2.56 \sqrt{n}) \]

		\begin{center}
		\begin{tabular}{c|c@{\ }c@{\ }c@{\ }c@{\ }c@{\ }c@{\ }c@{\ }c@{\ }c@{\ }c@{\ }c@{\ }c@{\ }c}
			$n$    & 0 & 1 & 2 & 3 & 4 & 5 & 6  & 7  & 8  & 9  & 20  & 50  & 100 \\ \hline
			$p(n)$ & 1 & 1 & 2 & 3 & 5 & 7 & 11 & 15 & 22 & 30 & 627 & $\mathtt{\sim}$2e5 & $\mathtt{\sim}$2e8 \\
		\end{tabular}
		\end{center}

	\subsubsection{Lucas' Theorem}
		Let $n,m$ be non-negative integers and $p$ a prime. Write $n=n_kp^k+...+n_1p+n_0$ and $m=m_kp^k+...+m_1p+m_0$. Then $\binom{n}{m} \equiv \prod_{i=0}^k\binom{n_i}{m_i} \pmod{p}$.

\subsection{Sum of Binomials (FFT)}
    Goal: Given freq. array $C$, compute $Ans[k] = \sum_{i} C[i] \binom{i}{k}$ for all $k$.
    Rewrite: $Ans[k] = \frac{1}{k!} \sum_{i} (C[i] \cdot i!) \frac{1}{(i-k)!}$.
    \begin{itemize}
        \item Construct $P$ where $P[i] = C[i] \cdot i!$
        \item Construct $Q$ where $Q[i] = (i!)^{-1}$
        \item Reverse $Q$ (to handle the $i-k$ subtraction).
        \item Multiply $R = NTT(P, Q)$.
        \item Result: $Ans[k] = R[k + |Q| - 1] \cdot \frac{1}{k!}$.
    \end{itemize}        
\subsection{General purpose numbers}
	\subsubsection{Bernoulli numbers}
		EGF of Bernoulli numbers is $B(t)=\frac{t}{e^t-1}$ (FFT-able).
		$B[0,\ldots] = [1, -\frac{1}{2}, \frac{1}{6}, 0, -\frac{1}{30}, 0, \frac{1}{42}, \ldots]$

		Sums of powers:
		\small
		\[ \sum_{i=1}^n n^m = \frac{1}{m+1} \sum_{k=0}^m \binom{m+1}{k} B_k \cdot (n+1)^{m+1-k} \]
		\normalsize

		Euler-Maclaurin formula for infinite sums:
		\small
		\[ \sum_{i=m}^{\infty} f(i) = \int_m^\infty f(x) dx - \sum_{k=1}^\infty \frac{B_k}{k!}f^{(k-1)}(m) \]
		\[ \approx \int_{m}^\infty f(x)dx + \frac{f(m)}{2} - \frac{f'(m)}{12} + \frac{f'''(m)}{720} + O(f^{(5)}(m)) \]
		\normalsize

	\subsubsection{Stirling numbers of the first kind}
		Number of permutations on $n$ items with $k$ cycles.
		\begin{align*}
			&c(n,k) = c(n-1,k-1) + (n-1) c(n-1,k),\ c(0,0) = 1 \\
			&\textstyle \sum_{k=0}^n c(n,k)x^k = x(x+1) \dots (x+n-1)
		\end{align*}
		$c(8,k) = 8, 0, 5040, 13068, 13132, 6769, 1960, 322, 28, 1$ \\
		$c(n,2) = 0, 0, 1, 3, 11, 50, 274, 1764, 13068, 109584, \dots$

	\subsubsection{Eulerian numbers}
		Number of permutations $\pi \in S_n$ in which exactly $k$ elements are greater than the previous element. $k$ $j$:s s.t. $\pi(j)>\pi(j+1)$, $k+1$ $j$:s s.t. $\pi(j)\geq j$, $k$ $j$:s s.t. $\pi(j)>j$.
		$$E(n,k) = (n-k)E(n-1,k-1) + (k+1)E(n-1,k)$$
		$$E(n,0) = E(n,n-1) = 1$$
		$$E(n,k) = \sum_{j=0}^k(-1)^j\binom{n+1}{j}(k+1-j)^n$$

	\subsubsection{Stirling numbers of the second kind}
		Partitions of $n$ distinct elements into exactly $k$ groups.
		$$S(n,k) = S(n-1,k-1) + k S(n-1,k)$$
		$$S(n,1) = S(n,n) = 1$$
		$$S(n,k) = \frac{1}{k!}\sum_{j=0}^k (-1)^{k-j}\binom{k}{j}j^n$$

	\subsubsection{Bell numbers}
		Total number of partitions of $n$ distinct elements. $B(n) =$
		$1, 1, 2, 5, 15, 52, 203, 877, 4140, 21147, \dots$. For $p$ prime,
		\[ B(p^m+n)\equiv mB(n)+B(n+1) \pmod{p} \]

	\subsubsection{Labeled unrooted trees}
		\begin{itemize}[noitemsep]
		\item on $n$ vertices: $n^{n-2}$
		\item on $k$ existing trees of size $n_i$: $n_1n_2\cdots n_k n^{k-2}$
		\item with degrees $d_i$: $(n-2)! / ((d_1-1)! \cdots (d_n-1)!)$
		\end{itemize}
		\vspace{1em} 

	\subsubsection{Catalan numbers}
		\[ C_n=\frac{1}{n+1}\binom{2n}{n}= \binom{2n}{n}-\binom{2n}{n+1} = \frac{(2n)!}{(n+1)!n!} \]
		\[ C_0=1,\ C_{n+1} = \frac{2(2n+1)}{n+2}C_n,\ C_{n+1}=\sum C_iC_{n-i} \]
		${C_n = 1, 1, 2, 5, 14, 42, 132, 429, 1430, 4862, 16796, 58786, \dots}$
		\begin{itemize}[noitemsep]
			\item sub-diagonal monotone paths in an $n\times n$ grid.
			\item strings with $n$ pairs of parenthesis, correctly nested.
			\item binary trees with with $n+1$ leaves (0 or 2 children).
			\item ordered trees with $n+1$ vertices.
			\item ways a convex polygon with $n+2$ sides can be cut into triangles by connecting vertices with straight lines.
			\item permutations of $[n]$ with no 3-term increasing subseq.
		\end{itemize}


\section{Number Theory}

\subsection{Bézout's identity}
For $a \neq $, $b \neq 0$, then $d=gcd(a,b)$ is the smallest positive integer for which there are integer solutions to
$$ax+by=d$$
If $(x,y)$ is one solution, then all solutions are given by
$$\left(x+\frac{kb}{\gcd(a,b)}, y-\frac{ka}{\gcd(a,b)}\right), \quad k\in\mathbb{Z}$$

\subsection{Primes}
	$p=962592769$ is such that $2^{21} \mid p-1$, which may be useful. For hashing
	use 970592641 (31-bit number), 31443539979727 (45-bit), 3006703054056749
	(52-bit). There are 78498 primes less than 1\,000\,000.

	Primitive roots exist modulo any prime power $p^a$, except for $p = 2, a > 2$, and there are $\phi(\phi(p^a))$ many.
	For $p = 2, a > 2$, the group $\mathbb Z_{2^a}^\times$ is instead isomorphic to $\mathbb Z_2 \times \mathbb Z_{2^{a-2}}$.

\subsection{Estimates}
	$\sum_{d|n} d = O(n \log \log n)$.

	The number of divisors of $n$ is at most around 100 for $n < 5e4$, 500 for $n < 1e7$, 2000 for $n < 1e10$, 6700 for $n < 1e12$, 200\,000 for $n < 1e19$.

\subsection{Mobius Function}
\[
	\mu(n) = \begin{cases} 0 & n \textrm{ is not square free}\\ 1 & n \textrm{ has even number of prime factors}\\ -1 & n \textrm{ has odd number of prime factors}\\\end{cases}
\]
  Mobius Inversion:
  \[ g(n) = \sum_{d|n} f(d) \Leftrightarrow f(n) = \sum_{d|n} \mu(d)g(n/d) \]
  Other useful formulas/forms:

  $ \sum_{d | n} \mu(d) = [ n = 1] $ (very useful)

  $ g(n) = \sum_{n|d} f(d) \Leftrightarrow f(n) = \sum_{n|d} \mu(d/n)g(d)$

 $ g(n) = \sum_{1 \leq m \leq n} f(\left\lfloor\frac{n}{m}\right \rfloor ) \Leftrightarrow f(n) = \sum_{1\leq m\leq n} \mu(m)g(\left\lfloor\frac{n}{m}\right\rfloor)$

 \subsection{Theorems}
    \textbf{Goldbach's conjecture:}
    Every even integer $n>2$ can be written as $n = a + b$ with $a,b$ prime.

    \textbf{Legendre's conjecture:}
    There is always at least one prime between $n^2$ and $(n+1)^2$.

    \textbf{Lagrange's four-square theorem:}
    Every positive integer can be written as
    \[
    n = a^2 + b^2 + c^2 + d^2.
    \]

    \textbf{Zeckendorf's theorem:}
    Every integer $n\ge 1$ has a unique representation as a sum of
    non-consecutive Fibonacci numbers:
    \[
    n = F_{i_1} + F_{i_2} + \dots + F_{i_k}, 
    \quad i_j - i_{j+1} \ge 2.
    \]

    \textbf{Euclid's formula (primitive Pythagorean triples):}
    The Pythagorean triples are uniquely generated by
    \[ a=k\cdot (m^{2}-n^{2}),\ \,b=k\cdot (2mn),\ \,c=k\cdot (m^{2}+n^{2}), \]
    with $m > n > 0$, $k > 0$, $m \bot n$, and either $m$ or $n$ even.

    \textbf{Wilson's theorem:}
    $n$ is prime iff
    \[
    (n-1)! \equiv -1 \pmod{n}.
    \]

    \textbf{Chicken McNugget theorem:}
    For coprime $n,m$, the largest integer not representable as
    $a n + b m$ (with $a,b\ge0$) is
    \[
    nm - n - m.
    \]
    There are $\frac{(n-1)(m-1)}{2}$ non-representable integers, and 
    for each pair $(k,\; nm-n-m-k)$ exactly one is representable.


\section{Graphs}
    \subsection{Flows and Matching}
    \subsubsection{Hall's Theorem}
        In bipartite graphs, there exists a perfect matching covering the entire side $X$
        if and only if for every subset $Y \subseteq X$,
        \[
        |Y| \le |N(Y)|,
        \]
        where $N(Y)$ denotes the set of neighbors of $Y$.

    \subsubsection{K\H{o}nig's Theorem}
        In a bipartite graph, the size of a Minimum Vertex Cover
        is equal to the size of a Maximum Matching.
        A Minimum Vertex Cover is a minimum set of vertices such that every edge
        of the graph has at least one endpoint in the set.

        As a consequence,
        \[
        n - \text{Maximum Matching} = \text{Maximum Independent Set},
        \]
        where a Maximum Independent Set is the largest set of vertices
        with no edges between them.

        \paragraph{Recovering the Minimum Vertex Cover}
        Given a maximum matching in a bipartite graph $(X, Y)$:
        \begin{itemize}
            \item Construct the residual graph by orienting:
            \begin{itemize}
                \item non-matching edges from $X$ to $Y$;
                \item matching edges from $Y$ to $X$.
            \end{itemize}
            \item Perform a BFS or DFS starting from all free (unmatched) vertices in $X$.
            \item Let $Z_X$ be the set of reachable vertices in $X$,
                and $Z_Y$ the set of reachable vertices in $Y$.
        \end{itemize}

        The Minimum Vertex Cover is given by:
        \[
        (X \setminus Z_X) \cup Z_Y.
        \]

    \subsubsection{Node-Disjoint Path Cover}
        A node-disjoint path cover is a set of paths such that each vertex
        belongs to exactly one path.

        In a directed acyclic graph (DAG),
        \[
        \text{Minimum Node-Disjoint Path Cover} = n - \text{Maximum Matching}.
        \]

        The construction is as follows:
        for each vertex $u$, create a copy $u'$.
        Add an edge $u \to v'$ if there exists an edge $u \to v$ in the original graph.

        \paragraph{Recovering the Paths}
        \begin{itemize}
            \item Vertices that do not appear as destinations in the matching
                are starting points of paths.
            \item Each matching edge $u \to v'$ corresponds to an edge $u \to v$
                in the original DAG.
            \item Following these edges reconstructs all paths of the path cover.
        \end{itemize}

    \subsubsection{General Path Cover}
        A general path cover is a path cover where a vertex may belong
        to more than one path.

        In a DAG, the construction is similar to the node-disjoint case,
        but an edge $u \to v'$ exists if there is a path from $u$ to $v$
        in the original graph.

        \paragraph{Recovering the Cover}
        The vertices can be grouped according to the edges used in the matching
        to form the path cover.

    \subsubsection{Dilworth's Theorem}
        An antichain is a set of vertices such that there is no path
        between any pair of vertices in the set.

        In a directed acyclic graph,
        \[
        \text{Minimum General Path Cover} = \text{Maximum Antichain}.
        \]

        \paragraph{Recovering a Maximum Antichain}
        Given a minimum general path cover, selecting one vertex from each path
        produces a maximum antichain.

	\subsection{Number of Spanning Trees}
		% I.e. matrix-tree theorem.
		% Source: https://en.wikipedia.org/wiki/Kirchhoff%27s_theorem
		% Test: stress-tests/graph/matrix-tree.cpp
		Create an $N\times N$ matrix \texttt{mat}, and for each edge $a \rightarrow b \in G$, do
		\texttt{mat[a][b]--, mat[b][b]++} (and \texttt{mat[b][a]--, mat[a][a]++} if $G$ is undirected).
		Remove the $i$th row and column and take the determinant; this yields the number of directed spanning trees rooted at $i$
		(if $G$ is undirected, remove any row/column).

	\subsection{Erdős–Gallai theorem}
		% Source: https://en.wikipedia.org/wiki/Erd%C5%91s%E2%80%93Gallai_theorem
		% Test: stress-tests/graph/erdos-gallai.cpp
		A simple graph with node degrees $d_1 \ge \dots \ge d_n$ exists iff $d_1 + \dots + d_n$ is even and for every $k = 1\dots n$,
		\[ \sum _{i=1}^{k}d_{i}\leq k(k-1)+\sum _{i=k+1}^{n}\min(d_{i},k). \]

    \subsection{Planar Graphs}
        If $G$ has $k$ connected components, then $n - m + f = k + 1$.

\section{Optimization tricks}
	\subsection{Bit hacks}
		\begin{itemize}
			\item \verb@for (int x = m; x; x = (x - 1)&m ) { ... }@ loops over all subset masks of \texttt{m} (except 0).
			\item \verb@c = x&-x, r = x+c; (((r^x) >> 2)/c) | r@ is the next number after \texttt{x} with the same number of bits set.
			\item \verb@rep(b,0,K) rep(i,0,(1 << K))@ \\ \verb@  if (i & 1 << b) D[i] += D[i^(1 << b)];@ computes all sums of subsets.
		\end{itemize}
	\subsection{Pragmas}
		\begin{itemize}
			\item \lstinline{#pragma GCC optimize ("Ofast")} will make GCC auto-vectorize loops and optimizes floating points better.
			\item \lstinline{#pragma GCC target ("avx2")} can double performance of vectorized code, but causes crashes on old machines.
			\item \lstinline{#pragma GCC target("bmi,bmi2,popcnt,lzcnt")} improve bit operations.
			\item \lstinline{#pragma GCC optimize("unroll-loops")} self explanatory.
		\end{itemize}

\section{Various}
\subsection{Master Theorem (Simple)}
    $T(n) = aT(n/b) + O(n^d)$. Compare $a$ vs $b^d$:
    \begin{itemize}
        \item $a > b^d \implies O(n^{\log_b a})$ (Work at leaves dominates)
        \item $a = b^d \implies O(n^d \log n)$ (Work is uniform)
        \item $a < b^d \implies O(n^d)$ (Work at root dominates)
    \end{itemize}